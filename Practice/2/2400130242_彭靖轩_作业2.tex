% !Mode:: "TeX:UTF-8"
% !TEX program  = xelatex
% Document class declaration
\documentclass{BlockChain}
% Import style package
\usepackage{BlockChain}
% Initialize document metadata variables
\sduCourse{区块链原理} % Course name and semester
\sduStudentId{202400130242} % Student ID
\sduName{彭靖轩} % Student name
\sduStudentClass{智能24} % Student class
\sduExperimentalTopics{Block} % Experiment topic
\sduDate{\today} % Current date
\sduTime{4小时}
% Begin document content
\begin{document}
\begin{sduDocument}
    \section{原理分析与步骤}
    \subsection{原理分析}
    本Python脚本通过解析标准输入数据,模拟Block结构,最终输出信息.
    \subsection{步骤分析}
    首先读取并解析标准输入数据,在给定genesis block的基础上,构造新的block(包括头和交易列表).其中区块头包括idx、时间戳、前一个区块的哈希和默克尔树根(参考实验1)等数据.
    \section{结论分析与体会}
    通过本次实验,掌握了区块链中区块结构的基本组成和数据处理方法.
    \sduAppendix
    \lstinputlisting[
        style = Python,
        caption = {\bf 源代码},
        label = {lst:Py},
    ]{main.py}
\end{sduDocument}
\end{document}
