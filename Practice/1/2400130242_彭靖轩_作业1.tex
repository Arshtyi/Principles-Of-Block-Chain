% !Mode:: "TeX:UTF-8"
% !TEX program  = xelatex

% Document class declaration
\documentclass{BlockChain}

% Import style package
\usepackage{BlockChain}

% Initialize document metadata variables
\sduCourse{区块链原理} % Course name and semester
\sduStudentId{202400130242} % Student ID
\sduName{彭靖轩} % Student name
\sduStudentClass{智能24} % Student class
\sduExperimentalTopics{Merkle Tree数据结构} % Experiment topic
\sduDate{\today} % Current date
\sduTime{4小时}
% Begin document content
\begin{document}
\begin{sduDocument}
\section{原理分析与步骤}
\subsection{原理分析}
本Python脚本实现简单的Merkle Tree,这是一种Hash Tree(这里使用SHA-256),用于高效和安全地验证数据的完整性.其叶结点的值是数据块的哈希值,非叶结点的值是其子结点哈希值的哈希值.通过这种结构,可以快速验证数据是否被篡改,而无需检查所有数据块.
\subsection{步骤分析}
通过原始数据块进行初始化,计算每个数据块的SHA-256哈希值作为叶结点.然后,通过迭代地将相邻节点的哈希值组合并计算其父节点的哈希值,逐层构建树结构.如果节点数为奇数,则复制最后一个节点以确保每层节点数为偶数.最终,根节点的哈希值代表整个数据集的完整性.

获取根结点的Hash值并验证,然后打印整棵树.
\section{结论分析与体会}
Merkle Tree是一种高效的数据结构,在区块链和分布式系统中广泛应用.通过这种树形结构,可以快速验证数据的完整性,而无需检查所有数据块.这不仅提高了验证效率,还增强了数据的安全性.在实际应用中,Merkle Tree有助于确保数据的一致性和防篡改性,是现代数字系统中的重要工具.
\sduAppendix
\lstinputlisting[
    style = Python,
    caption = {\bf 源代码},
    label = {lst:Py},
    ]{main.py}
\end{sduDocument}
\end{document}
