% !Mode:: "TeX:UTF-8"
% !TEX program  = xelatex
% Document class declaration
\documentclass{BlockChain}
% Import style package
\usepackage{BlockChain}
% Initialize document metadata variables
\sduCourse{区块链原理} % Course name and semester
\sduStudentId{202400130242} % Student ID
\sduName{彭靖轩} % Student name
\sduStudentClass{智能24} % Student class
\sduExperimentalTopics{Block} % Experiment topic
\sduDate{\today} % Current date
\sduTime{6小时}
% Begin document content
\begin{document}
\begin{sduDocument}
    \section{原理分析与步骤}
    \subsection{原理分析}
    本实验实现了基于工作量证明(Proof of Work, PoW)的区块链系统。核心原理:
    \begin{itemize}
        \item[1.] 每个区块包含区块头(索引、时间戳、前块哈希、Merkle根、nonce)和交易列表。
        \item[2.] 通过计算密集型任务寻找nonce值p,使得$hash(p \times p') $的前2位为"00",其中$p'$是前一区块的proof。
        \item[3.] 设置交易池容量为4,当交易数达到阈值时自动触发挖矿,将交易打包成区块。
        \item[4.] 使用交易列表的哈希值作为Merkle根,确保交易数据完整性。
    \end{itemize}
    \subsection{步骤分析}
    \begin{itemize}
        \item[1.] 初始化区块链,创建创世区块(索引0,无交易)。
        \item[2.] 添加交易到交易池,记录发送方、接收方和金额。
        \item[3.] 交易池满时,计算Merkle根,执行PoW算法寻找有效nonce。
        \item[4.] 创建新区块,包含区块头和交易列表,添加到区块链。
        \item[5.] 重复2-4,成功创建3个有效区块。
        \item[6.] 打印完整区块链,展示所有区块信息和交易记录。
    \end{itemize}
    \section{结论分析与体会}
    通过本次实验,成功实现了基于PoW的区块链系统,其优势在于:
    \begin{itemize}
        \item[1.] 理解了区块链的基本结构和工作原理
        \item[2.] 掌握了PoW算法的实现细节
        \item[3.] 体验了区块链数据的不可篡改性
    \end{itemize}
    \sduAppendix
    \lstinputlisting[
        style = Python,
        caption = {\bf 源代码},
        label = {lst:Py},
    ]{main.py}
\end{sduDocument}
\end{document}
